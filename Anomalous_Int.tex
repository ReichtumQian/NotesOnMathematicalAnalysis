
\chapter{反常积分}

一般的定积分都定义于闭区间中,
但很多实际问题中的积分可能区间无穷或者某点不连续,
这就需要引出无穷积分和瑕积分的概念。

\section{反常积分理论}

\subsection{无穷积分与瑕积分概念}

\begin{definition}[无穷积分与瑕积分]
  \begin{itemize}
  \item 无穷积分:
    设$f$在$[a,+\infty)$有定义,若极限$\lim \limits _{b \rightarrow +\infty}\int^b_af(x)\mathrm{d}x$存在,
    则记其为$\int^{+\infty}_a f(x)\mathrm{d}x$。
  \item 瑕积分:
    $f$在$(a,b]$有定义,且$\lim \limits _{x \rightarrow a^+}f(x) = \infty$,
    若$\lim \limits _{\epsilon \rightarrow 0^+}\int_{a + \epsilon}^b f(x)\mathrm{d}x $收敛,
    则记其为$\int_a^b f(x) \mathrm{d}x$,
  \end{itemize}
\end{definition}

\begin{note}
  如果$\lim \limits _{x \rightarrow a^+}f(x) \neq \infty$,则不能称$a$为$f(x)$的瑕点,
  比如$\int_0^1\frac{x}{1 - e^x}\mathrm{d}x$,$0$不是其瑕点,其就是个定积分。
\end{note}

\begin{theorem}[反常积分Leibniz公式]
  \begin{itemize}
  \item 无穷积分:
    设$\int^{+\infty}_af(x)\mathrm{d}x$存在,
    且$f$有原函数$F$,
    则$\int^{+\infty}_af(x)\mathrm{d}x = F(+\infty) - F(a)$
  \item 瑕积分:
    若$F$是$f$在$(a,b]$的一个原函数,
    $a$是瑕点,
    则$\int _a^bf(x)\mathrm{d}x = F(b) - F(a + 0)$
  \end{itemize}
\end{theorem}

\begin{theorem}[定义判敛]
  \begin{itemize}
  \item 无穷积分:$\int_a^{+\infty}f(x)\mathrm{d}x$收敛当且仅当$\forall \epsilon, \exists A, \forall u \geq A$,
    $|\int_u^{+\infty}f(x)\mathrm{d}x| < \epsilon$
  \item 瑕积分:$\int_a^bf(x)\mathrm{d}x$收敛当且仅当$\forall \epsilon, \exists \delta, \forall u \in (a,a + \delta)$有
    $|\int_a^u f(x)\mathrm{d}x| < \epsilon$
  \end{itemize}
\end{theorem}

\begin{proof}
  根据收敛的定义可知。
\end{proof}

\begin{theorem}[$\frac{1}{x^p}$的反常积分]
  $\int_0^1 \frac{\mathrm{d} x}{x^p}$在$p < 1$时收敛,$p \geq 1$时发散。
  $\int_1^{+\infty} \frac{\mathrm{d} x}{x^p}$在$p > 1$时收敛,$p \leq 1$时发散
\end{theorem}

\begin{proof}
  只讨论$[1,+\infty)$上的,直接积分:
  \begin{equation*}
    \int_1^u \frac{\mathrm{d} x}{x^p} =
    \begin{cases}
      \frac{1}{1-p} (u^{1-p}-1), & p \neq 1\\
      \ln u, & p = 1
    \end{cases} \Rightarrow \lim \limits _{u \rightarrow +\infty}\int_1^u \frac{\mathrm{d} x}{x^p} =
    \begin{cases}
      \frac{1}{p-1}, & p > 1\\
      +\infty , & p \leq 1
    \end{cases}
  \end{equation*}
\end{proof}

\subsection{反常积分比较判别法}

\begin{theorem}[Cauchy收敛准则]
  \begin{itemize}
  \item 无穷积分:
    $\int^{\infty}_a f(x)\mathrm{d}x$收敛当且仅当$\forall \epsilon, \exists G \geq a, \forall u_1,u_2 > G$,满足
    \begin{equation*}
       |\int_{u_1}^{u_2}f(x)\mathrm{d}x| < \epsilon
    \end{equation*}
  \item 瑕积分:
    $\int_a^bf(x)\mathrm{d}x$的瑕点为$a$,则其收敛当且仅当$\forall \epsilon, \exists \delta, \forall u_1,u_2 \in (a,a+\delta)$满足
    \begin{equation*}
      |\int_{u_1}^{u_2}f(x)\mathrm{d}x| < \epsilon
    \end{equation*}
  \end{itemize}
\end{theorem}

\begin{definition}[绝对收敛与条件收敛]
  若$\int_a^b|f(x)|\mathrm{d}x$收敛($b$可以为$\infty$),则称$\int_a^bf(x)\mathrm{d}x$绝对收敛;
  若$\int_a^bf(x)\mathrm{d}x$收敛但不绝对收敛,则称其条件收敛。
\end{definition}

\begin{theorem}[绝对收敛]
  \begin{itemize}
  \item 无穷积分:
    若$\int_a^{+\infty}|f(x)|\mathrm{d}x$收敛,则$\int^{+\infty}_af(x)\mathrm{d}x$也收敛,
    且有下述等式
    \begin{equation*}
      |\int_a^{+\infty}f(x)\mathrm{d}x| \leq \int^{+\infty}_a |f(x)|\mathrm{d}x
    \end{equation*}
  \item 瑕积分:
    设$f$瑕点为$a$,$f$在$(a,b]$任一内闭区间上可积,
    则当$\int_a^b|f(x)|\mathrm{d}x$收敛时,$\int_a^bf(x)\mathrm{d}x$也收敛,
    且
    \begin{equation*}
      |\int_a^b f(x)\mathrm{d}x| \leq \int_a^b |f(x)|\mathrm{d}x
    \end{equation*}
  \end{itemize}
\end{theorem}

\begin{proof}
  用Cauchy收敛准则转换至有限区间,再根据积分绝对值放缩可得到结论。
\end{proof}

\begin{theorem}[比较判别法]
  下面考虑$f(x)$ \textbf{非负},则有
  \begin{itemize}
  \item 无穷积分:
    $f,g$在任意有限区间可积,且$0 \leq f(x) \leq g(x)$,
    当$\int_a^{+\infty}g(x)\mathrm{d}x$收敛时,$\int^{+\infty}_af(x)\mathrm{d}x$必收敛
  \item 瑕积分:
    $f,g$在内包有限闭区间可积,且$0 \leq f(x) \leq g(x)$,
    当$\int_a^bg(x)\mathrm{d}x$收敛时,$\int^b_af(x)\mathrm{d}x$必收敛
  \end{itemize}
\end{theorem}

\begin{corollary}[等价无穷小判别]
  \begin{itemize}
  \item 无穷积分:
    $f$在任意有限闭区间可积,且$x \rightarrow \infty$时$f(x) \sim \frac{\lambda}{x^p}$,
    若$p > 1$则收敛,$p \leq 1$则发散
  \item 瑕积分:
    $f$在任意闭区间可积,$x \rightarrow a$时$f(x) \sim \frac{\lambda}{(x - a)^p}$,
    若$p < 1$则收敛,$p \geq 1$则发散
  \end{itemize}
\end{corollary}

\begin{exercise}[基础训练]
  判断(1)$\int_0^{+\infty}\frac{x \arctan x}{1 + x^p}\mathrm{d}x$
  (2)$\int_0^{+\infty}\frac{1}{x^p + x^q}\mathrm{d} x$
  (3)$\int_0^{+\infty}\frac{x^p}{1 + x^q}\mathrm{d}x$
\end{exercise}

\begin{solution}
  (1)
  $0$不是瑕点,只需考虑$\infty$。
  若$p > 0$,$\frac{x \arctan x}{1 + x^p} \sim \frac{\pi}{2}\frac{1}{x^{p-1}}$,因此要求$p > 2$,此时$x$不是瑕点。
  若$p = 0$,显然发散。
  若$p < 0$,$\infty$显然发散。

  (2)$0,+\infty$均有可能是瑕点。
  先考虑$p = q$,此时$\frac{1}{x^p+x^q} = \frac{1}{2x^p}$,
  $\infty$处要求$p > 1$,但是$0$处要求$p < 1$,因此不可能成立。
  若$p < q$,则$x \rightarrow 0$时$\frac{1}{x^p + x^q} \sim \frac{1}{x^p}$,要求$p < 1$,
  同理$x \rightarrow \infty$时要求$q > 1$

  (3)$\frac{x^p}{1 + x^q} = \frac{1}{x^{-p} + x^{q - p}}$,
  当$-p < 1 < q - p$或者$q-p < 1 < -p$时收敛,其余情况发散。
\end{solution}

\begin{corollary}[对数、指数比较判别法]
  对$\forall \alpha > 0$,$\lim \limits _{x \rightarrow 0^+}x^{\alpha}\ln x= 0, \lim \limits _{x \rightarrow \infty} \frac{\ln x}{x^{\alpha}} = 0, \lim \limits _{x \rightarrow +\infty}x^{\alpha}e^{-x} = 0$:
  \begin{itemize}
  \item $x \rightarrow \infty$时,等价视为$\ln x \sim x^{\epsilon}, e^x \sim x^{+\infty}$
  \item $x \rightarrow 0$时,$\ln (1+x) \sim x, e^x - 1 \sim x, \ln x \sim \frac{1}{x^{\epsilon}}, e^x \sim 1$
  \end{itemize}
\end{corollary}

~

\begin{exercise}[对数、指数比较判别训练]
  判断收敛性:(1)$\int_0^{+\infty}\frac{x}{1 - e^x}\mathrm{d}x$
  (2)$\int_0^{+\infty}\frac{\ln(1 + x)}{x^p}\mathrm{d}x$
  (3)$\int_0^{+\infty}\frac{\ln x}{e^x}\mathrm{d}x$(重点!)
\end{exercise}

\begin{solution}
  (1)由于$0$处极限为$1$,不是瑕点。$\infty$处$e^x \sim x^{\infty}$,因此$\lim \limits _{x \rightarrow \infty}x^2 \frac{x}{1 - e^x} = 0$,根据比较判别法可知收敛

  (2)$0$处等价于$\frac{1}{x^{p-1}}$,要求$p < 2$,
  $+\infty$处等价于$\frac{1}{x^p}$,要求$p > 1$,
  因此$1 < p < 2$时收敛

  (3)$\lim \limits _{x \rightarrow 0}\frac{\ln x \cdot x^{1/2}}{e^x} = 0$,因此$0$处收敛。
  $\infty$处$\lim \limits _{x \rightarrow \infty}x^2 \frac{\ln x}{e^x} = 0$,因此$\infty$处也收敛。
\end{solution}

\begin{exercise}[对数比较判别进阶]
  判断$\int_3^{+\infty}\frac{\mathrm{d}x}{x^p (\ln x)^q (\ln \ln x)^r}$的敛散性
\end{exercise}

\begin{solution}
  $p > 1$肯定收敛,$p < 1$肯定发散。
  若$p = 1$,则积分变为$\int_3^{+\infty} \frac{\mathrm{d}(\ln x)}{(\ln x)^q (\ln \ln x)^r}$,要求$q > 1$。
  若$p = q = 1$,则同理要求$r > 1$。
\end{solution}

\begin{exercise}[变号情况:绝对收敛+比较判别法]
  判断敛散性:(1)$\int_1^{+\infty}\frac{\cos x \sin \frac{1}{x}}{x}\mathrm{d}x$
\end{exercise}

\begin{solution}
  (1)$|\frac{\cos x \sin \frac{1}{x}}{x}| \sim \frac{1}{x^2}$,因此绝对收敛
\end{solution}

\begin{exercise}[三角双瑕点训练]
  重点:$|q| < 1$,判断$\int _0^{\pi}\frac{\sin^{p-1}x}{|1 + q \cos x|^p}\mathrm{d}x$的敛散性
\end{exercise}

\begin{solution}
  $0$处$\frac{\sin^{p-1}x}{|1 + q \cos x|^p} \sim \frac{x^{p-1}}{|1 + q|^p}$,要求$p - 1<1$。
  $\pi$处$\frac{\sin^{p-1}x}{|1 + q \cos x|^p} = \frac{\sin^{p-1}(\pi - x)}{|1 + q\cos x|^p} \sim \frac{(\pi - x)^{p-1}}{|1 + q|^p}$,
  要求$p - 1 < 1$
\end{solution}

\subsection{Dirichlet与Abel判别法}

\begin{theorem}[Dirichlet与Abel判别法]
  \begin{itemize}
  \item Dirichlet判别法:若(1)$F(x) = \int_a^x f(t)\mathrm{d}t$在$[a,+\infty)$有界(2)$g(x)$在$[a,+\infty)$上当$x \rightarrow \infty$时单调趋于$0$,则$\int_a^{+\infty}f(x)g(x)\mathrm{d}x$收敛。
  \item Abel判别法:若(1)$\int_a^{+\infty}f(x)\mathrm{d}x$收敛(2)$g(x)$单调有界,则$\int_a^{+\infty}f(x)g(x)\mathrm{d}x$收敛
  \end{itemize}
\end{theorem}

~

\begin{exercise}[在$\sin x$反常积分判敛的应用]
  判断以下带三角函数的反常积分的绝对收敛性与条件收敛性:
  \begin{enumerate}
  \item $\int _1^{+\infty} \frac{\sin x}{x^p}\mathrm{d}x, \int_1^{+\infty}\frac{\cos x}{x^p}\mathrm{d}x$:$p>0$就收敛,$p >1$绝对收敛,$0 < p \leq  1$条件收敛
  \item $\int_0^{+\infty}\frac{\sin x}{x^p}\mathrm{d} x$:$1 < p < 2$时绝对收敛,$0 < p \leq 1$时条件收敛,其余发散
  \item $ \int_0^{+\infty}\frac{\cos x}{x^p}\mathrm{d} x$:$1 < p < 2$时绝对收敛,$0 < p \leq 1$时条件收敛,其余发散
  \item 换元:$\int _0^{+\infty} \sin x^2 \mathrm{d}x, \int_0^{+\infty}\cos x^2 \mathrm{d}x, \int_0^{+\infty}x \sin x^4 \mathrm{d}x, \int_0^1 \frac{1}{x^p} \sin \frac{1}{x} \mathrm{d} x$
  \item $\int_0^{+\infty} x^p \sin x^q \mathrm{d} x$
  \item $\int_0^{+\infty} \frac{\sin x}{x^p + x^q}\mathrm{d} x$
  \end{enumerate}
\end{exercise}

\begin{proof}
  (1)$p > 1$时$|\frac{\sin x}{x^p}| \leq \frac{1}{x^p}$因此绝对收敛。
  $0 < p \leq 1$时根据Dirichlet判别法可知收敛,
  下证不绝对收敛,根据$|\sin x| \geq \sin^2 x$,
  得到
  \begin{equation*}
    \left| \frac{\sin x}{x^p} \right| \geq \frac{\sin^2x}{x^p} = \frac{1}{2} \left( \frac{1}{x^p} - \frac{\cos 2x}{x^p} \right)
  \end{equation*}
  而$\int_1^{+\infty}\frac{1}{x^p}\mathrm{d}x$不收敛,故$\int_1^{+\infty} \left| \frac{\sin x}{x^p} \right|$在$0 < p \leq 1$时不收敛。
  $p \leq 0$时,取$u_1 = 2k\pi, u_2 = 2k\pi + \frac{\pi}{2}$,
  则
  \begin{equation*}
    \left| \int_{u_1}^{u_2} \frac{\sin x}{x^p}\mathrm{d} x \right| \geq \int_{u_1}^{u_2} \sin x\mathrm{d}x = 1
  \end{equation*}
  显然发散。

  (2)分两段,这里考虑$\int_0^1 \frac{\sin x}{x^p}\mathrm{d} x$,
  当$x \rightarrow 0$时$\frac{\sin x}{x^p} \sim \frac{1}{x^{p-1}}$,
  因此$p < 2$时收敛且绝对收敛,
  $p \geq 2$时发散。
  结合(1)得到$1 < p < 2$时绝对收敛,$0 < p \leq 1$时条件收敛,其余发散

  (3)$\frac{\cos x}{x^p} \sim \frac{1}{x^p}$,
  结合(1)得到$0 < p < 1$时条件收敛,其余发散。

  (4)前三个做$t = x^2$换元,
  第一个变成$\frac{1}{2}\int_0^{+\infty} \frac{\sin t}{\sqrt{t}}  \mathrm{d} t$,因此条件收敛。
  第二个$\int_0^{+\infty} \cos x^2 \mathrm{d} x$同理条件收敛。
  第三个换元得到$\frac{1}{2}\int _1^{+\infty}\sin t^2 \mathrm{d}t$,变成第一个,故条件收敛。
  第四个做$t = \frac{1}{x}$换元得到$\int_{+\infty}^1 -t^{p-2} \sin t \mathrm{d} t = \int_1^{+\infty} \frac{\sin t}{t^{2-p}}\mathrm{d} t$,
  $2-p > 1$时绝对收敛,$0 < 2-p \leq 1$条件收敛。

  (5)$q = 0$时显然无法保证$0,+\infty$同时收敛,原因在于$0$要求$p < 1$,$\infty$要求$p > 1$。
  $q > 0$时,令$t = x^q$,得到
  \begin{equation*}
    \int_0^{+\infty}x^p \sin x^q \mathrm{d} x = \frac{1}{q} \int_0^{+\infty} \frac{\sin t}{t^{1- \frac{p+1}{q}}}\mathrm{d} t
  \end{equation*}
  转换为前面的问题。
\end{proof}

\begin{note}
  $\int_0^{+\infty}\frac{\sin x}{x^p}\mathrm{d} x, \int_0^{+\infty} \frac{\cos x}{x^p}\mathrm{d} x$的收敛情况要当作结论背下来。
\end{note}

~

\begin{exercise}[三角有理式敛散性判别]
  判断如下几道三角函数有理式的敛散性(难度较高):

  (1)判断$\int_0^{+\infty} \frac{\sin x}{2x + 3 \sin x} \mathrm{d} x$的敛散性

  (2)判断$\int_a^{+\infty} \frac{\sin x}{x^p + \sin x}\mathrm{d} x$的敛散性

  (3)证明$\int_0^{+\infty} \frac{\mathrm{d} x}{1 + x^2 \sin x^2}\mathrm{d} x$发散
\end{exercise}

\begin{solution}
  (1)该积分没有瑕点,只需要考虑$\infty$,其敛散性等价于$\int_2^{\infty} \frac{\sin x}{2x + 3 \sin x}\mathrm{d} x$。
  这里必须拆项(不能直接放缩原因在于分母阶数不够)
  \begin{equation*}
    \frac{\sin x}{2x + 3 \sin x} =\frac{2x \sin x}{2x(2x + 3\sin x)} = \frac{(2x + 3 \sin x)\sin x - 3\sin^2x}{2x(2x + 3\sin x) } =  \frac{\sin x}{2x} - \frac{3\sin^2 x}{2x(2x + 3\sin x)}
  \end{equation*}
  考虑前者,显然$\frac{1}{2x}$单调递减趋于$0$,而
  \begin{equation*}
    \left| \int_2^u \sin \mathrm{d} x  \right| = \left| \cos 2 - \cos u \right| \leq 2
  \end{equation*}
  根据Dirichlet判别法知前者收敛。后者先进行放缩
  \begin{equation*}
    0 \leq \frac{3 \sin^2 x}{2x(2x + 3 \sin x)} \leq \frac{3}{2x(2x - 3)}
  \end{equation*}
  在无穷处等价于$\frac{1}{x^2}$,根据比较原则得到收敛。综上收敛
\end{solution}


\subsection{函数在无穷远处的性质}

\noindent 一、若极限存在,则必为$0$

~

\begin{exercise}
  $f(x)$在$[a,u]$可积,$\int_a^{+\infty}f(x)\mathrm{d}x$收敛,
  若$\lim \limits _{x \rightarrow +\infty}f(x) = A$,则$A = 0$
\end{exercise}

\begin{proof}
  若$A \neq 0$,不妨设$A > 0$,则$\exists M \geq a, \forall x \geq M, f(x) \geq \frac{A}{2}$,
  而$\int_a^{+\infty}\frac{A}{2}\mathrm{d}x$发散,根据比较判别法可知$\int_a^{+\infty}f(x)\mathrm{d}x$发散
\end{proof}

~

\begin{exercise}
  $f(x)$在$[a,+\infty)$上连续可微,
  且$\int_a^{+\infty}f(x)\mathrm{d}x,\int_a^{+\infty}f^{\prime}(x)\mathrm{d}x$收敛,则$\lim \limits _{x \rightarrow +\infty}f(x) = 0$
\end{exercise}

\begin{proof}
  由于$\int_a^{+\infty}f^{\prime}(x)\mathrm{d}x$收敛,因此$\lim \limits _{u \rightarrow +\infty}\int_a^uf^{\prime}(x)\mathrm{d}x = \lim \limits _{u \rightarrow +\infty}f(u) - f(a)$存在,
  根据极限存在可知极限为$0$(前一题结论)。
\end{proof}

~

\begin{exercise}
  $f(x)$在$[a,+\infty)$上连续,$\int_a^{+\infty}f(x)\mathrm{d}x$收敛,则存在递增趋于$+\infty$的数列$x_n$,
  使得$\lim \limits _{n \rightarrow \infty} f(x_n) = 0$
\end{exercise}

\begin{proof}
  根据Cauchy收敛准则,取$\epsilon_1 = 1, \exists A_1^{\prime}, A_1^{\prime\prime} = A_1^{\prime} + 1$,
  使得$|\int_{A_{1}^{\prime}}^{A_1^{\prime\prime}}f(x)\mathrm{d}x| < \epsilon_1 = 1$。
  同理对$\epsilon_2 = \frac{1}{2}, \exists A_2^{\prime} \geq A_1^{\prime\prime}, A_2^{\prime\prime} = A_2^{\prime} + 1$,使得$|\int_{A_2^{\prime}}^{A_2^{\prime\prime}}f(x)\mathrm{d}x| < \epsilon_2 = \frac{1}{2}$。
  以此类推,
  根据积分第一中值定理
  $\exists x_i \in [A_i^{\prime},A_i^{\prime\prime}]$使得$|f(x_n)| = |\int_{A_n^{\prime}}^{A_n^{\prime\prime}}f(x)\mathrm{d}x|$,
  显然$x_n$单增,$|f(x_n)| < \frac{1}{n} \rightarrow 0$。
\end{proof}

~

\noindent 二、极限不一定存在:光满足非负、连续、可导等并不能说明极限存在,甚至不一定有界

~

\begin{exercise}
  举例说明$f(x)$在任意有限区间$[a,u]$非负可积,
  $\int_a^{+\infty}f(x)\mathrm{d}x$收敛,则$f(x)$不一定有界。
\end{exercise}

\begin{solution}
  $f(x) =
  \begin{cases}
    x, &x\text{为整数}\\
    0,&x\text{不为整数}
  \end{cases}
  $,
  如果要求恒正则取$g(x) = f(x) + e^{-x}$
\end{solution}

~

\begin{exercise}
  举例说明$f(x)$为$[a,+\infty)$上连续函数,
  $\int_a^{+\infty}f(x)\mathrm{d}x$收敛,不一定有$\lim \limits _{x \rightarrow +\infty}f(x) = 0$
\end{exercise}

\begin{solution}
  例如$ \int_1^{+\infty}\sin x^2 \mathrm{d}x$,令$t = x^2$以及Dirichlet判别法可知收敛,
  但是极限不为$0$。
\end{solution}

~

\begin{exercise}
  举例说明$[a,+\infty)$恒正的连续可微函数,$\int_a^{+\infty}f(x)\mathrm{d}x$收敛,则$f(x)$不一定有界
\end{exercise}

\begin{solution}
  $\int_1^{+\infty}\frac{x}{1 + x^6\sin^2 x}\mathrm{d}x$
\end{solution}

~

\noindent 三、获得极限存在:虽然非负、连续、可导不能推出$\lim \limits _{x \rightarrow +\infty}f(x) = 0$,
但一致连续和单调可以推出该性质。

~

\begin{exercise}[经典老题:一致连续]
  $f(x)$在$[a,+\infty)$上一致连续,$\int_a^{+\infty}f(x)\mathrm{d}x$收敛,则$\lim \limits _{x \rightarrow +\infty}f(x) = 0$
\end{exercise}

\begin{proof}
  反设$\lim \limits _{x \rightarrow +\infty}f(x) \neq 0$,则$\exists \epsilon_0$以及趋于无穷的$x_n$使得
  $|f(x_n)| \geq \epsilon_0$,
  对上述$\epsilon_0$,根据一致连续$\exists \delta,\forall x^{\prime},x^{\prime\prime} \in [a,+\infty)$只要
  $|x^{\prime} - x^{\prime\prime}| \leq \delta$,就有$|f(x^{\prime}) - f(x^{\prime\prime})| < \frac{\epsilon_0}{2}$。
  对每个$x_n$,当$x \in [x_n,x_n+\delta]$时,$|f(x) - f(x_n)| < \frac{\epsilon_0}{2}$,
  从而$f(x)$在$[x_n,x_n+\delta]$不变号,且
  \begin{equation*}
    |f(x)| \geq |f(x_n)| - |f(x) - f(x_n)| > \frac{\epsilon_0}{2}
  \end{equation*}
  对$\forall M > a, \exists N, x_N > M$满足$|\int_{x_N}^{x_N + \delta} f(x)\mathrm{d}x| = \int_{x_N}^{x_N + \delta}|f(x)|\mathrm{d}x \geq \int_{x_N}^{x_N + \delta}\frac{\epsilon_0}{2}\mathrm{d}x = \frac{\epsilon_0\delta}{2}$不趋于$0$,
  根据Cauchy收敛准则逆否命题可知$\int_a^{+\infty}f(x)\mathrm{d}x$发散,矛盾。
\end{proof}

~

\begin{exercise}[单调]
  $f(x)$在$[a,+\infty)$单调,且$\int_a^{+\infty}f(x)\mathrm{d} x$收敛,则$\lim \limits _{x \rightarrow +\infty}f(x) = 0 $,
  且$\lim \limits _{x \rightarrow +\infty}xf(x) = 0$。
\end{exercise}

\begin{note}
  反之不一定,即使有$\lim \limits _{x \rightarrow +\infty}xf(x) = 0$,也不一定有$\int_a^{+\infty}f(x)\mathrm{d} x$收敛,
  例如$f(x) = \frac{1}{x \ln x}$
\end{note}

~

\section{反常积分的计算}

\subsection{两个常用的基础反常积分}


\begin{exercise}
  计算$\int_0^{+\infty} e^{-ax}\cos bx \mathrm{d} x, \int_0^{+\infty} e^{-ax} \sin bx \mathrm{d} x$,其中$a > 0$
\end{exercise}

\begin{solution}
  直接写不定积分即可(具体计算见配对积分法一节),
  $I_1 = \frac{e^{-ax}}{a^2 + b^2} (a \cos bx + b \sin bx) \bigg|_0^{+\infty} = \frac{a}{a^2+b^2}$,
  同理得到$I_2 = \frac{b}{a^2 + b^2}$
\end{solution}

~

\begin{exercise}[对数配对积分]
  (1)计算$I_n = \int_0^1 (\ln x)^n \mathrm{d} x$
  (2)计算$I(m,n) = \int_0^1 x^m (\ln x)^n \mathrm{d} x$
\end{exercise}

\begin{solution}
  (1)做一次分部积分得到$I_n = x (\ln x)^n \bigg|_0^1 - \int_0^1 n (\ln x)^{n-1}\mathrm{d}x = -n I_{n-1}$,
  而$I_1 = \int_0^1 \ln x \mathrm{d} x = x \ln x \bigg|_0^1 - x \bigg|_0^1 = -1$,
  综上得到$I_n = (-1)^nn!$

  (2)$I(m,n) = - \frac{n}{m+1}I(m,n-1)$,
  结合$I(m,0) = \frac{1}{m+1}$可得$I(m,n) = \frac{(-1)^nn!}{(m+1)^{n+1}}$
\end{solution}

\subsection{富如兰尼积分}

\begin{theorem}[两侧极限存在则无穷积分收敛]
  重点:$f(x)$在任意有限区间可积,
  $\lim \limits _{x \rightarrow +\infty}f(x) = A, \lim \limits _{x \rightarrow -\infty}f(x) = B$,
  给定$a \in \mathbb{R}$,则下面积分收敛,且值为$a(A - B)$
  \begin{equation*}
    \int_{-\infty}^{+\infty}[f(a+x) - f(x)]\mathrm{d} x
  \end{equation*}
\end{theorem}

\begin{proof}
  根据定义,$\int_{-\infty}^{+\infty}f(x)\mathrm{d} x = \lim \limits _{M \rightarrow +\infty}\lim \limits _{m \rightarrow -\infty} \int _m^M [f(a+x) - f(x)]\mathrm{d} x$,
  对有限区域积分做如下变换:
  \begin{equation*}
    I = \int_m^M [f(a+x) - f(x)]\mathrm{d} x = \int_{m+a}^{M+a}f(x)\mathrm{d}x - \int_m^M f(x)\mathrm{d} x = \int_M^{M+a}f(x)\mathrm{d} x - \int_m^{m+a}f(x)\mathrm{d}x
  \end{equation*}
  代入$A,B$,得到$I = \int_M^{M+a}[f(x) - A]\mathrm{d}x - \int _m^{m+a}[f(x) - B]\mathrm{d} x + a(A-B)$,
  根据极限$\lim \limits _{M \rightarrow +\infty} [f(x) - A]\mathrm{d} x = 0$,
  同理$\lim \limits _{N \rightarrow +\infty}[f(x) - B]\mathrm{d} x = 0$,
  因此取极限得到收敛,且值为$a(A - B)$
\end{proof}

~

\begin{theorem}[富如兰尼积分]
  $f(x)$为$[0,+\infty)$连续函数,$a,b>0$,则
  \begin{enumerate}
  \item 若$\lim \limits _{x \rightarrow +\infty}f(x)$存在,则
    \begin{equation*}
      \int_0^{+\infty} \frac{f(ax) - f(bx)}{x} \mathrm{d} x = [f(0) - f(+\infty)] \ln \frac{b}{a}
    \end{equation*}
  \item 若$\exists A > 0$,使得$\int_A^{+\infty} \frac{f(x)}{x}\mathrm{d} x$收敛,则
    \begin{equation*}
      \int_0^{+\infty} \frac{f(ax) - f(bx)}{x}\mathrm{d} x = f(0) \ln \frac{b}{a}
    \end{equation*}
  \end{enumerate}
\end{theorem}

\begin{proof}
  下面对$I = \int_m^M \frac{f(ax) - f(bx)}{x}\mathrm{d} x$进行化简:
  \begin{equation*}
    I = \int_m^M \frac{f(ax)}{x}\mathrm{d} x - \int _m^M \frac{f(bx)}{x}\mathrm{d} x = \int_{am}^{aM}\frac{f(x)}{x}\mathrm{d} x - \int_{bm}^{bM} \frac{f(x)}{x}\mathrm{d} x = \int _{am}^{bm} \frac{f(x)}{x}\mathrm{d} x - \int_{aM}^{bM}\frac{f(x)}{x}\mathrm{d} x
  \end{equation*}

  (1)根据积分第一中值定理得到
  \begin{equation*}
    \begin{cases}
      \int_{am}^{bm} \frac{f(x)}{x} \mathrm{d} x = f(\xi) \int_{am}^{bm} \frac{1}{x}\mathrm{d} x = f(\xi) \ln \frac{b}{a}\\
      \int_{aM}^{bM} \frac{f(x)}{x}\mathrm{d} x = f(\eta) \int_{aM}^{bM} \frac{1}{x}\mathrm{d} x = f(\eta) \ln \frac{b}{a}
    \end{cases}
  \end{equation*}
  当$m \rightarrow 0^+, M \rightarrow +\infty$时可代入极限,
  因此结果为$[f(0) - f(+\infty)] \ln \frac{b}{a}$

  (2)由于$\int_A^{+\infty}f(x)\mathrm{d} x$收敛,
  因此第二项$\lim \limits _{M \rightarrow +\infty} \frac{f(x)}{x}\mathrm{d} x = 0$,
  同时第一项用积分中值定理取极限得到$f(0) \ln \frac{b}{a}$
\end{proof}

~

\begin{exercise}[富如兰尼积分的应用]
  计算以下反常积分

  (1)$\int_0^{+\infty} \frac{\arctan ax - \arctan bx}{x}\mathrm{d} x$
  (2)$\int_0^{+\infty} \frac{e^{-ax} - e^{-bx}}{x}\mathrm{d} x$
  (3)$\int_0^{+\infty} \frac{\cos ax - \cos bx}{x}\mathrm{d} x$
\end{exercise}

\begin{solution}
  (1)$\arctan 0 = 0, \arctan (+\infty) = \frac{\pi}{2}$,
  因此积分为$- \frac{\pi}{2} \ln \frac{b}{a}$

  (2)$e^0 = 1, e^{- \infty} = 0$,
  因此积分为$\ln \frac{b}{a}$

  (3)$\cos 0 = 1, \int _A^{+\infty} \frac{\cos x}{x}\mathrm{d} x$收敛,
  因此积分为$\ln \frac{b}{a}$
\end{solution}


~


\subsection{Dirichlet积分}

\begin{lemma}
  设$p > 0$,则含参积分$\int_0^{+\infty} e^{-px} \frac{\sin bx - \sin ax}{x} \mathrm{d} x = \arctan \frac{b}{p} - \arctan \frac{a}{p}$
\end{lemma}

\begin{proof}
  首先$\frac{\sin bx - \sin ax}{x} = \int_a^b \cos xy \mathrm{d} y$,
  因此
  \begin{equation*}
    I = \int_0^{+\infty} \mathrm{d} x \int _a^b e^{-px}\cos xy \mathrm{d} y
  \end{equation*}
  由于$|e^{-px}\cos xy| \leq e^{-px}$,显然$\int_0^{+\infty} e^{-px}\mathrm{d} x$绝对收敛,
  因此$\int_0^{+\infty}e^{-px}\cos xy\mathrm{d}x, y \in [a,b]$一致收敛,交换积分顺序得到
  \begin{align*}
    I = \int_a^b \mathrm{d} y \int_0^{+\infty} e^{-px} \cos xy \mathrm{d} x
  \end{align*}
  根据表格法得到:
  \begin{equation*}
    \int_0^{+\infty} e^{-px}\cos xy \mathrm{d} x = - \frac{1}{p} e^{-px}\cos xy   \bigg|^{+\infty}_0 + \frac{y}{p^2} e^{-px} \sin xy   \bigg|_0^{+\infty} - \int_0^{+\infty} \frac{y^2}{p^2} e^{-px} \cos xy\mathrm{d} x 
  \end{equation*}
  因此得到$\frac{y^2 + p^2}{p^2} \int _0^{+\infty} \cos xy\mathrm{d} y = \frac{1}{p}$
  \begin{equation*}
    I = \int_a^b \frac{p}{p^2 + y^2}\mathrm{d} y = \arctan \frac{b}{p} - \arctan \frac{a}{p}
  \end{equation*}
\end{proof}

\begin{theorem}[Dirichlet积分]
  积分$\int _0^{+\infty} \frac{\sin x}{x} \mathrm{d} x = \frac{\pi}{2}$,
  进一步地,$\int_0^{+\infty} \frac{\sin px}{x}\mathrm{d} x = \frac{\pi}{2} \text{sgn}(p)$
\end{theorem}

\begin{proof}
  令引理中$a = 0$,将$p$看成参数,$G(p)= \int_0^{+\infty} e^{-px} \frac{\sin bx}{x} \mathrm{d} x = \arctan \frac{b}{p}$,
  由于$\int_0^{+\infty} \frac{\sin bx}{x}\mathrm{d} x$条件收敛,
  $e^{-px}$在固定$x$时关于$p$单调,且一致有界,
  根据Abel判别法可知$G(p)$在$[0,1]$上一致收敛,
  从而
  \begin{equation*}
    G(0) = \lim \limits _{p \rightarrow 0^+} G(p) = \lim \limits _{p \rightarrow 0^+} \arctan \frac{b}{p} = \frac{\pi}{2}\text{sgn}(b)
  \end{equation*}
  特别地,$b = 1$得到$\int_0^{+\infty} \frac{\sin x}{x}\mathrm{d} x = \frac{\pi}{2}$
\end{proof}

~

\begin{exercise}[Dirichlet积分的应用]
  计算以下反常积分:
  (1)重点:$\int_0^{+\infty} \frac{\sin x^2}{x}\mathrm{d} x$
  (2)重点:$\int_0^{+\infty} \frac{\sin^2 x}{x^2}\mathrm{d} x$

  (3)$\int_0^{+\infty} \frac{\sin^4 x}{x^2}\mathrm{d} x$
  (4)$\int_0^{+\infty} \frac{\sin^4 x}{x^4}\mathrm{d} x$
  (5)$\int_0^{+\infty} \frac{x - \sin x}{x^3}\mathrm{d} x$
\end{exercise}

\begin{solution}
  (1)做$t = x^2$,得到$\frac{1}{2}\int_0^{+\infty} \frac{\sin t}{t}\mathrm{d} t = \frac{\pi}{4}$

  (2)$\int_0^{+\infty} \frac{\sin^2 x}{x^2}\mathrm{d} x = - \int_0^{+\infty} \sin^2 x \mathrm{d} \left( \frac{1}{x} \right)$,
  用分部积分得到
  \begin{equation*}
    I = - \frac{\sin^2 x}{x} \bigg|_0^{+\infty} + \int_0^{+\infty} \frac{\sin 2x}{2x}\mathrm{d} 2x = \frac{\pi}{2}
  \end{equation*}

  (3)先分部积分得到
  \begin{equation*}
    \int_0^{+\infty} \frac{\sin^4 x}{x^2}\mathrm{d} x = - \int_0^{+\infty} \sin^4 x \mathrm{d} \left( \frac{1}{x} \right) = \int_0^{+\infty} \frac{4 \sin^3 x \cos x}{x}\mathrm{d} x
  \end{equation*}
  根据$4\sin^3x \cos x = \sin 2x(1 - \cos 2x) = \sin 2x - \frac{1}{2} \sin 4x$得到
  \begin{equation*}
    I = \int_0^{+\infty} \frac{\sin 2x}{2x}\mathrm{d}(2x) - \frac{1}{2} \int_0^{+\infty} \frac{\sin 4x}{4x}\mathrm{d} (4x) = \frac{\pi}{4}
  \end{equation*}
\end{solution}
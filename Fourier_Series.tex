
\chapter{傅里叶级数}

\section{以$2\pi$为周期的傅里叶级数}

\begin{definition}[$2\pi$周期傅里叶展开]
  对于$2\pi$周期函数$f(x)$,定义其傅里叶展开为$f(x) \sim \frac{a_0}{2} + \sum\limits_{n = 1}^{\infty}(a_n \cos nx + b_n \sin nx)$,
  其中
  \begin{equation*}
    \begin{cases}
      a_n = \frac{1}{\pi} \int_{-\pi}^{\pi}f(x) \cos nx \mathrm{d}x & n = 0,1,2,\cdots\\
      b_n = \frac{1}{\pi} \int_{-\pi}^{\pi}f(x) \sin nx \mathrm{d}x & n = 1,2,\cdots
    \end{cases}
  \end{equation*}
\end{definition}

\begin{note}
  计算时一定要注意第一项是$\frac{a_0}{2}$!
\end{note}

\begin{theorem}[收敛定理]
  若以$2\pi$为周期的函数$f$在$[-\pi, \pi]$上按段光滑,
  则对$\forall x \in [-\pi,\pi]$,傅里叶级数收敛于$f$在$x$左右极限的算术平均值。
\end{theorem}

\begin{corollary}[连续特殊情况]
  若$f$是以$2\pi$为周期的连续函数,
  且在$[-\pi,\pi]$上按段光滑,则$f$的傅里叶级数在$(-\infty,+\infty)$上收敛于$f$
\end{corollary}

~

\begin{exercise}[几个经典展开]
  (1)求$f(x) = x^2$在$(-\pi, \pi)$上的傅里叶级数,
  并求$\sum\limits_{n = 1}^{\infty} \frac{1}{n^2}$
\end{exercise}

\begin{solution}
  (1)计算得到$a_0 = \frac{2}{3}\pi^2$,
  $a_n = \frac{1}{\pi} \int_{-\pi}^{\pi}x^2 \cos nx \mathrm{d} x = (-1)^n \frac{4}{n^2}$,
  因此得到
  \begin{equation*}
    x^2 = \frac{1}{3}\pi^2 + \sum\limits_{n = 1}^{\infty} (-1)^n \frac{4}{n^2}
  \end{equation*}
  令$x = \pi$得到
  \begin{equation*}
    \pi^2 = \frac{1}{3}\pi^2 + \sum\limits_{n = 1}^{\infty} \frac{4}{n^2} \Rightarrow \sum\limits_{n = 1}^{\infty} \frac{1}{n^2} = \frac{\pi^2}{6}
  \end{equation*}
\end{solution}

\begin{note}
  注意$\cos n\pi = (-1)^n$。
\end{note}




~



\section{以$2l$为周期的傅里叶级数}

\begin{definition}[$2l$周期的傅里叶级数]
  $f(x)$以$2l$为周期,则其傅里叶级数为$f(x) \sim \frac{a_0}{2} + \sum\limits_{n = 1}^{\infty}(a_n \cos \frac{n\pi x}{l} + b_n \sin \frac{n \pi x}{l})$,其中
  \begin{equation*}
    \begin{cases}
      a_n = \frac{1}{l}\int _{-l}^l f(x) \cos \frac{n \pi x}{l}\mathrm{d}x, & n = 0,1,2,\cdots\\
      b_n = \frac{1}{l}\int_{-l}^l f(x) \sin \frac{n \pi x}{l} \mathrm{d}x, & n = 1,2,\cdots
    \end{cases}
  \end{equation*}
\end{definition}


\section{Parseval等式}

\begin{theorem}[Parseval等式]
  $\frac{1}{\pi} \int_{-\pi}^{\pi} [f(x)]^2 \mathrm{d} x = \frac{a_0^2}{2} + \sum\limits_{n = 1}^{\infty} (a_n^2 + b_n^2)$
\end{theorem}

\begin{exercise}[计算数项级数]
  (1)对$f(x) = x^2$展开,并证明$\sum\limits_{n = 1}^{\infty} \frac{1}{n^4} = \frac{\pi^4}{90}$
\end{exercise}




